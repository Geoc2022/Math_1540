\documentclass[12pt]{article}

%%%%%%%%%%%%%%%%%%%%%%
%%%%%%% PACAKAGES %%%%%%%
%%%%%%%%%%%%%%%%%%%%%


%%%%%%%%%%%%% GRAPHICS/FONTS
\usepackage{amsmath,amsfonts,graphicx,tikz-cd,pgfplots}

%%%%%%%%%%%%%% FORMATTING
\usepackage{geometry,titlesec,hyperref,xhfill,setspace,float,fancyhdr,xifthen,lastpage}

%%%%%%%%%TIKZ STUFF
\usetikzlibrary{positioning,calc,decorations.markings,shapes.geometric}
\usepgfplotslibrary{polar,fillbetween}

%%%%%%FOR THE GRAPHS
\pgfplotsset{my style/.append style={axis x line=middle, axis y line=
middle, xlabel={$x$}, ylabel={$y$}, axis equal }}

%%%%%%MARGINS
\geometry{
a4paper,
left=0.25in,
right=0.25in,
top=0.25in,
bottom=0.75in
}

%%%%%%MAKES THE HEADER AND FOOTER FANCY
\fancyhf{}
\renewcommand{\headrulewidth}{0pt}
\pagestyle{fancy}

%%%%PAGENUMBERS_________REMEMBER TO CHANGE THIS REGULARLY!!!!!!
\cfoot{Page \thepage/\pageref*{LastPage}}








%%%%%%%%%%%%%%%%%%%%%%
%%%%%%% COMMANDS %%%%%%%
%%%%%%%%%%%%%%%%%%%%%
\newcommand{\underscore}{\underline{\hspace{2mm}}}
\newcommand{\Z}{\mathbb{Z}}
\newcommand{\Q}{\mathbb{Q}}
\newcommand{\C}{\mathbb{C}}
\newcommand{\R}{\mathbb{R}}
\newcommand{\Gal}{\text{Gal}}
\newcommand{\Aut}{\text{Aut}}
\newcommand{\GL}{\text{GL}}
\newcommand{\PGL}{\text{PGL}}
\newcommand{\PSL}{\text{PSL}}
\newcommand{\End}{\text{End}}
\newcommand{\lra}{\longrightarrow}

\newcommand{\del}{\partial}
\newcommand{\RP}{\mathbb{R}P}
\newcommand{\K}{\emph{K}}

%TEXT COMMAND
\newcommand{\T}[1][]{\text{#1}}
\newcommand{\TB}[1][]{\mathbb{#1}}

\newcommand{\xlra}[1][]{%
  \ifthenelse{\isempty{#1}}%
    {\xrightarrow{\phantom{,,,,,,}}}% if #1 is empty
    {\xrightarrow{\phantom{,,}#1\phantom{,,}}}% if #1 is not empty
}


%%%%%%%%%%%%%%%%%%%%%%%%%%%%%%%%%%%%%%%%
%%%%%%%%BEGINNING OF ACTUAL DOCUMENT %%%%%%%%%%
%%%%%%%%%%%%%%%%%%%%%%%%%%%%%%%%%%%%%%%%







%%%%TITLE______REMEMBER TO REGULARLY CHANGE THIS!!!!
\title{Math 1540 Spring 2023 - Homework 3}
\date{}

\begin{document}
\maketitle
\vspace{-0.5in}
%%%%%%%%%%%%%%%%
\begin{spacing}{1.5}
\noindent \textbf{Instructions:}  This assignment is worth twenty points.  Please complete the following problems assigned below.  Submissions with insufficient explanation may lose points due to a lack of reasoning or clarity.  If you are handwriting your work, please ensure it is readable and well-formatted for the grader.\\
\\
Be sure when uploading your work to \textbf{assign problems to pages}.  Problems with pages not assigned to them \textbf{may not be graded}.  
\end{spacing}




%%%%%%%%%%%%%%%%%%
\vspace{10mm}\noindent
\textbf{Textbook Problems: }72, 76, 77
\\

\noindent
\textbf{Additional Problems:}  We're largely working in fields, so you can typically assume $E,F,B$ are fields unless otherwise stated.\\
\\
1.  Let $\alpha \in E/F$.  Prove that $F(\alpha) = \{f(\alpha)/g(\alpha) \in E \, | \, f,g \in F[x]\text{ and } g(\alpha) \neq 0\}$.  \\
\\
2.  Let $f(x) \in \Z_{2}[x]$ be defined by $f(x) = x^4+x+1$.  Let $E$ be a splitting field of $f(x)$.  Calculate, with proof, $[E:\Z_{2}]$. \\
\\
3.  Consider $\Q(\sqrt{2+\sqrt{3}},\sqrt{2-\sqrt{3}})/\Q$.  Is this extension simple?  If so, find an element that generates the extension, along with its minimal polynomial.  Calculate the degree of this extension.   \\
\\
4.  Let $x^{2} + x + 2 \in \Z_{5}[x]$ and consider a splitting field $E$.  As observed in class, $E$ is a vector space over $\Z_{5}$.  Using a root of $f(x)$, construct a faithful representation of $\Z_{5}^{\times} \lra \GL(2,\Z_{5})$.  Write out the image of your representation.   
%%%%%%%%
%%%%%%%%% 
\end{document}