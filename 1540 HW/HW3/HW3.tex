\documentclass{article}

% basics
\usepackage[utf8]{inputenc}
\usepackage[T1]{fontenc}
\usepackage{textcomp}
\usepackage{url}
\usepackage{hyperref}
\hypersetup{
    colorlinks,
    linkcolor={bb_g},
    citecolor={bb_g},
    urlcolor={bb_g}
}
\usepackage{graphicx}
\usepackage{float}
\usepackage{booktabs}
\usepackage{enumitem}
% \usepackage{parskip}
\usepackage{emptypage}
\usepackage{subcaption}
\usepackage{multicol}
\usepackage[usenames,dvipsnames]{xcolor}

% \usepackage{cmbright}

\usepackage{amsmath, amsfonts, mathtools, amsthm, amssymb}
\usepackage[left=1.5in,right=1.5in,top=1.5in,bottom=1.5in]{geometry}
\usepackage{mathrsfs}
\usepackage{cancel}
\usepackage{bm}
\newcommand\N{\ensuremath{\mathbb{N}}}
\newcommand\R{\ensuremath{\mathbb{R}}}
\newcommand\Z{\ensuremath{\mathbb{Z}}}
\renewcommand\O{\ensuremath{\emptyset}}
\newcommand\Q{\ensuremath{\mathbb{Q}}}
\newcommand\C{\ensuremath{\mathbb{C}}}
\renewcommand{\qedsymbol}{$\blacksquare$}
\DeclareMathOperator{\sgn}{sgn}
\usepackage{systeme}
\let\svlim\lim\def\lim{\svlim\limits}
\let\implies\Rightarrow
\let\impliedby\Leftarrow
\let\iff\Leftrightarrow
\let\epsilon\varepsilon
\usepackage{stmaryrd} % for \lightning
\newcommand\contra{\scalebox{1.1}{$\lightning$}}
\let\phi\varphi


% correct
\definecolor{correct}{HTML}{009900}
\newcommand\correct[2]{\ensuremath{\:}{\color{red}{#1}}\ensuremath{\to }{\color{correct}{#2}}\ensuremath{\:}}
\newcommand\green[1]{{\color{correct}{#1}}}


% Nord colors
\definecolor{d_0}{HTML}{2E3440}
\definecolor{d_1}{HTML}{3B4252}
\definecolor{d_2}{HTML}{434C5E}
\definecolor{d_3}{HTML}{4C566A}

\definecolor{w_0}{HTML}{D8DEE9}
\definecolor{w_1}{HTML}{E5E9F0}
\definecolor{w_2}{HTML}{ECEFF4}

\definecolor{b_ggg}{HTML}{8FBCBB}
\definecolor{b_gg}{HTML}{88C0D0}
\definecolor{b_g}{HTML}{81A1C1}
\definecolor{bb_g}{HTML}{5E81AC}

\definecolor{a_red}{HTML}{BF616A}
\definecolor{a_orange}{HTML}{D08770}
\definecolor{a_yellow}{HTML}{EBCB8B}
\definecolor{a_green}{HTML}{A3BE8C}
\definecolor{a_purple}{HTML}{B48EAD}


% Font
% \renewcommand{\familydefault}{\sfdefault}


% horizontal rule
\newcommand\hr{
    \noindent\rule[0.5ex]{\linewidth}{0.5pt}
}


% hide parts
\newcommand\hide[1]{}


% si unitx
\usepackage{siunitx}
\sisetup{locale = FR}
% \renewcommand\vec[1]{\mathbf{#1}}
\newcommand\mat[1]{\mathbf{#1}}


% tikz
\usepackage{tikz}
\usepackage{tikz-cd}
\usetikzlibrary{intersections, angles, quotes, calc, positioning}
\usetikzlibrary{arrows.meta}
\usepackage{pgfplots}
\pgfplotsset{compat=1.13}


\tikzset{
    force/.style={thick, {Circle[length=2pt]}-stealth, shorten <=-1pt}
}


% theorems
\makeatother
\usepackage{thmtools}
\usepackage[framemethod=TikZ]{mdframed}
\mdfsetup{skipabove=1em,skipbelow=0em}


\theoremstyle{definition}

\newcommand{\declaretheoremstylebox}[2]{
    \declaretheoremstyle[
        headfont=\bfseries\sffamily\color{#1}, bodyfont=\normalfont,
        mdframed={
            linewidth=2pt,
            rightline=false, topline=false, bottomline=false,
            linecolor=#1, backgroundcolor=#1!5
        }
    ]{thm#2box}    
}

\newcommand{\declaretheoremstyleline}[2]{
    \declaretheoremstyle[
        headfont=\bfseries\sffamily\color{#1}, bodyfont=\normalfont,
        mdframed={
            linewidth=2pt,
            rightline=false, topline=false, bottomline=false,
            linecolor=#1
        }
    ]{thm#2line}    
}

\declaretheoremstylebox{b_ggg}{green}
\declaretheoremstylebox{b_g}{blue}
\declaretheoremstylebox{a_red}{red}
\declaretheoremstylebox{a_orange}{orange}
\declaretheoremstylebox{a_yellow!60!a_orange}{yellow}

\declaretheoremstyleline{bb_g}{blue}
\declaretheoremstyleline{b_gg}{green}

\declaretheoremstyle[
    headfont=\bfseries\sffamily\color{a_red}, bodyfont=\normalfont,
    numbered=no,
    mdframed={
        linewidth=2pt,
        rightline=false, topline=false, bottomline=false,
        linecolor=a_red, backgroundcolor=a_red!1,
    },
    qed=\qedsymbol
]{thmproofbox}

\declaretheoremstyle[
    headfont=\bfseries\sffamily\color{b_g}, bodyfont=\normalfont,
    numbered=no,
    mdframed={
        linewidth=2pt,
        rightline=false, topline=false, bottomline=false,
        linecolor=b_g, backgroundcolor=b_g!1,
    },
]{thmexplanationbox}

% \declaretheoremstyle[headfont=\bfseries\sffamily, bodyfont=\normalfont, mdframed={ nobreak } ]{thmgreenbox}
% \declaretheoremstyle[headfont=\bfseries\sffamily, bodyfont=\normalfont, mdframed={ nobreak } ]{thmredbox}
% \declaretheoremstyle[headfont=\bfseries\sffamily, bodyfont=\normalfont]{thmbluebox}
% \declaretheoremstyle[headfont=\bfseries\sffamily, bodyfont=\normalfont]{thmblueline}
% \declaretheoremstyle[headfont=\bfseries\sffamily, bodyfont=\normalfont, numbered=no, mdframed={ rightline=false, topline=false, bottomline=false, }, qed=\qedsymbol ]{thmproofbox}
% \declaretheoremstyle[headfont=\bfseries\sffamily, bodyfont=\normalfont, numbered=no, mdframed={ nobreak, rightline=false, topline=false, bottomline=false } ]{thmexplanationbox}

\declaretheorem[style=thmgreenbox, name=Definition]{definition}
\declaretheorem[style=thmbluebox, numbered=no, name=Example]{example}
\declaretheorem[style=thmorangebox, name=Proposition]{proposition}
\declaretheorem[style=thmredbox, name=Theorem]{theorem}
\declaretheorem[style=thmyellowbox, name=Lemma]{lemma}
\declaretheorem[style=thmredbox, numbered=no, name=Corollary]{corollary}

\declaretheorem[style=thmproofbox, name=Proof]{replacementproof}
\renewenvironment{proof}[1][\proofname]{\vspace{-10pt}\begin{replacementproof}}{\end{replacementproof}}

\declaretheorem[style=thmexplanationbox, name=Proof]{tmpexplanation}
\newenvironment{explanation}[1][]{\vspace{-10pt}\begin{tmpexplanation}}{\end{tmpexplanation}}

\declaretheorem[style=thmgreenline, numbered=no, name=Remark]{remark}
\declaretheorem[style=thmblueline, numbered=no, name=Note]{note}

\newtheorem*{uovt}{UOVT}
\newtheorem*{notation}{Notation}
\newtheorem*{previouslyseen}{As previously seen}
\newtheorem*{problem}{Problem}
\newtheorem*{observe}{Observe}
\newtheorem*{property}{Property}
\newtheorem*{intuition}{Intuition}


\usepackage{etoolbox}
\AtEndEnvironment{vb}{\null\hfill$\diamond$}%
\AtEndEnvironment{intermezzo}{\null\hfill$\diamond$}%
% \AtEndEnvironment{opmerking}{\null\hfill$\diamond$}%

% http://tex.stackexchange.com/questions/22119/how-can-i-change-the-spacing-before-theorems-with-amsthm
\makeatletter
% \def\thm@space@setup{%
%   \thm@preskip=\parskip \thm@postskip=0pt
% }


\newcommand{\exercise}[1]{%
    \def\@exercise{#1}%
    \subsection*{Exercise #1}
}

\newcommand{\subexercise}[1]{%
    \subsubsection*{Exercise \@exercise.#1}
}


\usepackage{xifthen}

% Notes
\usepackage{marginnote}
\let\marginpar\marginnote

\def\testdateparts#1{\dateparts#1\relax}
\def\dateparts#1 #2 #3 #4 #5\relax{
    \marginpar{\small\textsf{\mbox{#1 #2 #3 #5}}}
}

\def\@lecture{}%
\newcommand{\lecture}[3]{
	\ifthenelse{\isempty{#3}}{%
		\def\@lecture{Lecture #1}%
	}{%
		\def\@lecture{Lecture #1: #3}%
	}%
	\section*{\@lecture}
	\marginpar{\small\textsf{\mbox{#2}}}
}


% \renewcommand\date[1]{\marginpar{#1}}


% fancy headers
\usepackage{fancyhdr}
\pagestyle{fancy}

% LE: left even
% RO: right odd
% CE, CO: center even, center odd
\fancyhead[LE, RO]{George C.}

\fancyhead[RO, LE]{\@lecture} % Right odd,  Left even
\fancyhead[RE, LO]{}          % Right even, Left odd
\fancyfoot[RO, LE]{\thepage}  % Right odd,  Left even
\fancyfoot[RE, LO]{}          % Right even, Left odd
\fancyfoot[C]{\leftmark}     % Center

\makeatother


% figure support
\usepackage{import}
\usepackage{xifthen}
\pdfminorversion=7
\usepackage{pdfpages}
\usepackage{transparent}
\newcommand{\incfig}[1]{%
    \def\svgwidth{\columnwidth}
    \import{./figures/}{#1.pdf_tex}
}

% %http://tex.stackexchange.com/questions/76273/multiple-pdfs-with-page-group-included-in-a-single-page-warning
\pdfsuppresswarningpagegroup=1

\setcounter{section}{-1}

\title{Math 1540: HW1}

\begin{document}

\maketitle

\emph{Book Problems:}
\begin{enumerate}
    \item[72.] 
    \begin{enumerate}
        \item[(i)] Since \(\alpha\) and \(\beta\) are algebraic over \(F\) there exist monic irreducible polynomials \(p_1\) and \(p_2\) of finite degree with roots \(\alpha\) and \(\beta\) respectively. Therefore, we can say that \([F(\alpha) : F] = \deg(p_1)\) and \([F(\alpha, \beta) : F(\alpha)] = \deg(p_2)\) and using the degree formula we can see that \([F(\alpha, \beta) : F] = [F(\alpha, \beta) : F(\alpha)][F(\alpha) : F] = \deg(p_1) \cdot \deg(p_2)\). Moreover, this is a finite degree extension. Finally, we can use the fact if E / F is a finite extension, then it is an algebraic extension to show that every element of \(F(\alpha, \beta)\) (ex. \(\alpha + \beta\), \(\alpha \beta \), \(\alpha^{-1}\)) is algebraic over \(F\)
        \item[(ii)] 
        
        \(K\) contains \(F\): \(\forall x \in F\) \(x\) is algebraic over \(F\) so \(\forall x \in F\) \(x \in K\) 
        
        \(K\) is a subfield of \(E\): For all \(x\) in \(K\) but not \(F\) we know that from the prior result that \(x^{-1}\), \(x + y\) where \(y \in F\), \(xy\) where \(y \in F\), etc. are all algebraic. This shows that \(K\) is closed and contains additive/multiplicative inverses. Moreover, \(K\) has additive and multiplicative identities from \(F\) and gets associativity, commutativity, distributivity laws from \(E\). Therefore, \(K\) is a field. By definition \(K \subseteq E\). Therefore, \(K\) is a subfield of \(E\).

        \item[(iii)] \(\mathbb{A}/\mathbb{Q}\) is not finite since it must contain all roots of polynomials of the form \(x^2 - p\) where \(p\) is prime, which would require \(\mathbb{Q}\) have a nonfinite extension with \(\mathbb{Q} (\sqrt{2}, \sqrt{3} \ldots \sqrt{p} \ldots)\), and we know that there are an infinite number of primes.

    \end{enumerate} 
    \item[76.] Suppose that there exists some \(a \in F\) s.t. \(a\) does not have a \(p\)th root in \(F\). Let \(b^p = a\) in an extension of \(F\). Therefore, in \(F[x]\), there exists \(x^p - a = x^p - b^p = (x - b)^p\). If \(x^p - a\) is not irreducible over \(F\), then \((x - b)^i\) where \(1 \leq i \leq p - 1\) must be a factor in \(F[x]\) and all its coefficients must be in \(F\). \((x - b)^i = x^i - i b x^{i -1} \ldots \), so \(- i b \in F\) so \(b^p = a \in F\), but we assumed \(a\) does not have a \(p\)th root in \(F\) — contradiction. Therefore, \(x^p -a\) is irreducible. From here we see that irreducible polynomials must be inseparable, because if they weren't, they could be reduced by using arithmetic in characteristic \(p\) 

    \item[77.] Let \(F\) be a finite field of characteristic \(p\), \(\phi\) be the map from \(F\) to itself where \(\forall x \in F\) \(\phi (x) = x^p\)
    
    From this map it is evident:
    \begin{eqnarray*}
        \phi (x y) = (x y)^p = x^p y^p = \phi (x) \phi (y) \\
        \phi (x + y) = (x + y)^p = x^p + y^p
    \end{eqnarray*}
    
    Moreover, \(\phi\) is injective since \(x^p \neq 0\); therefore it is surjective (\(F\) is finite) and \(F^p = F\) — \(F\) is perfect
\end{enumerate}

\emph{Given Problems:}
\begin{enumerate}
    \item[1.] Let the field of fractions be written as \(Frac(F[x])=\{f(x) / g(x) \in E \mid f, g \in F[x] \text { and } g(x) \neq 0\}\). Let \(\phi\) be a map between \(F[x]\) and \(Frac(F[x])\) where \(\phi\) sends \(f(x) \in F[x]\) to \(f(x)/1\). This map is well defined since \(f(x) = g(x) \implies \phi(f(x)) = f(x)/1 = g(x)/1 = \phi(g(x))\). And injective since only \(0\) is sent to \(\phi(0) = 0/1\). Moreover, since \(F(x)\) is a field if \(x \in E/F\) we can construct the inverse map \(\sigma\) from \(Frac(F[x])\) to \(F[x]\) where \(\sigma\) sends \(f(x)/g(x)\) to \(f(x)g^{-1}(x) \in F[x]\). This map is well defined since \(f_1(x)/f_2(x) = g_1(x)/g_2(x) \implies \sigma (f_1(x)/f_2(x)) = f_1(x)f_2^{-1}(x) = g_1(x)g_2^{-1}(x) = \phi(g_1(x)/g_2(x)) \iff f_1(x)g_2(x) = g_1(x)f_2(x)\). Moreover, it is surjective since \(\sigma(f(x)/1) = f(x)\) for all \(f(x) \in F[x]\). Therefore, there is an isomorphism between \(Frac(F[x])\) and \(F(x)\)

    \item[2.] Although I could show that \(f(x)\) has no linear factors since \(f(0) \neq 0, f(1) \neq 0\) and no irreducible quadratic factor (of which there is one \(x^2 + x + 1\) and \((x^2 + x + 1)^2 = x^4 + x^2 + 1 \neq f(x)\)) (what I've written so far is sufficient for irreducibility), we can use Rabin's test:
    
    \emph{Rabin's test of Irreducibility:}

    Let $f(x)$ be a polynomial of degree $n$ over $\mathbb{F}_p$. Then $f$ is irreducible over $\mathbb{F}_p$ if and only if $f(x)$ divides $x^{p^n}-x$, and $\operatorname{gcd}\left(f(x), x^{p^{n / q}}-x\right)=1$ for each prime divisor $q$ of $n$.

    Here it is sufficient to show that \(\operatorname{gcd}(x^4 + x + 1, x^{2^{4}} - x)\) is a multiple of \(x^4 + x + 1\) and that \(\operatorname{gcd}(x^4 + x + 1, x^{2^{2}} - x) = 1\)
    
    I've computed both calculations in Mathmatica:
    \begin{eqnarray*}
        \operatorname{PolynomialExtendedGCD}[x^4 + x + 1, x^{2^4} - x, x, Modulus -> 2][[1]] = x^4 + x + 1 \\
        \operatorname{PolynomialExtendedGCD}[x^4 + x + 1, x^{2^{4/2}} - x, x, 
        Modulus -> 2][[1]] = 1
    \end{eqnarray*}

    Therefore, both criteria are satisfied, so \(x^4 + x + 1\) is irreducible in \(\mathbb{Z} _2\). Therefore, \(E\), the splitting field of \(x^4 + x + 1\), has the same degree of \(x^4 + x + 1\), \(4\).

    \item[3.] 
    We find that \(1/\sqrt{2 + \sqrt{3}} = \sqrt{2 - \sqrt{3}}\) from the following:
    \begin{eqnarray*}
        \frac{1}{\sqrt{2 - \sqrt{3}}} \\
        \frac{1}{\sqrt{2 - \sqrt{3}}} \frac{\sqrt{2 + \sqrt{3}}}{\sqrt{2 + \sqrt{3}}} \\
        \frac{\sqrt{2 + \sqrt{3}}}{\sqrt{2 - \sqrt{3}}\sqrt{2 + \sqrt{3}}} \\ 
        \frac{\sqrt{2 + \sqrt{3}}}{\sqrt{(2 - \sqrt{3}) (2 + \sqrt{3})}} \\
        \frac{\sqrt{2 + \sqrt{3}}}{\sqrt{1}} \\
        \sqrt{2 + \sqrt{3}}
    \end{eqnarray*}
    
    Therefore, \(\mathbb{Q} (\sqrt{2 + \sqrt{3}})\) will contain \(\sqrt{2 - \sqrt{3}}\) as the multiplicative inverse of \(\sqrt{2 + \sqrt{3}}\) and thus will contain \(\mathbb{Q} (\sqrt{2 + \sqrt{3}}, \sqrt{2 - \sqrt{3}})\), and it is evident that \(\mathbb{Q} (\sqrt{2 + \sqrt{3}}, \sqrt{2 - \sqrt{3}})\) contians \(\mathbb{Q} (\sqrt{2 + \sqrt{3}})\), so \(\mathbb{Q} (\sqrt{2 + \sqrt{3}}) = \mathbb{Q} (\sqrt{2 + \sqrt{3}}, \sqrt{2 - \sqrt{3}})\). 
    
    Therefore, the extension \(\mathbb{Q} (\sqrt{2 + \sqrt{3}}, \sqrt{2 - \sqrt{3}})\) is simple since it can be written as \(\mathbb{Q} (\sqrt{2 + \sqrt{3}})\) with generating element \(\sqrt{2 + \sqrt{3}}\) 

    To find the minimal polynomial of \(\sqrt{2 + \sqrt{3}}\) let's set \(x = \sqrt{2 + \sqrt{3}}\):
    \begin{eqnarray*}
        \sqrt{2 + \sqrt{3}} = x \\
        2 + \sqrt{3} = x^2 \\
        \sqrt{3} = x^2 - 2 \\
        3 = (x^2 - 2)^2 \\
        0 = (x^2 - 2)^2 - 3 \\
        0 = x^4 - 4x^2 + 1
    \end{eqnarray*}

    Since the degree of the extension is equal to the degree of the minimal polynomial, the degree of the extension is \(4\).
\end{enumerate}

\end{document}