\documentclass{article}

% basics
\usepackage[utf8]{inputenc}
\usepackage[T1]{fontenc}
\usepackage{textcomp}
\usepackage{url}
\usepackage{hyperref}
\hypersetup{
    colorlinks,
    linkcolor={bb_g},
    citecolor={bb_g},
    urlcolor={bb_g}
}
\usepackage{graphicx}
\usepackage{float}
\usepackage{booktabs}
\usepackage{enumitem}
% \usepackage{parskip}
\usepackage{emptypage}
\usepackage{subcaption}
\usepackage{multicol}
\usepackage[usenames,dvipsnames]{xcolor}

% \usepackage{cmbright}

\usepackage{amsmath, amsfonts, mathtools, amsthm, amssymb}
\usepackage[left=1.5in,right=1.5in,top=1.5in,bottom=1.5in]{geometry}
\usepackage{mathrsfs}
\usepackage{cancel}
\usepackage{bm}
\newcommand\N{\ensuremath{\mathbb{N}}}
\newcommand\R{\ensuremath{\mathbb{R}}}
\newcommand\Z{\ensuremath{\mathbb{Z}}}
\renewcommand\O{\ensuremath{\emptyset}}
\newcommand\Q{\ensuremath{\mathbb{Q}}}
\newcommand\C{\ensuremath{\mathbb{C}}}
\renewcommand{\qedsymbol}{$\blacksquare$}
\DeclareMathOperator{\sgn}{sgn}
\usepackage{systeme}
\let\svlim\lim\def\lim{\svlim\limits}
\let\implies\Rightarrow
\let\impliedby\Leftarrow
\let\iff\Leftrightarrow
\let\epsilon\varepsilon
\usepackage{stmaryrd} % for \lightning
\newcommand\contra{\scalebox{1.1}{$\lightning$}}
\let\phi\varphi


% correct
\definecolor{correct}{HTML}{009900}
\newcommand\correct[2]{\ensuremath{\:}{\color{red}{#1}}\ensuremath{\to }{\color{correct}{#2}}\ensuremath{\:}}
\newcommand\green[1]{{\color{correct}{#1}}}


% Nord colors
\definecolor{d_0}{HTML}{2E3440}
\definecolor{d_1}{HTML}{3B4252}
\definecolor{d_2}{HTML}{434C5E}
\definecolor{d_3}{HTML}{4C566A}

\definecolor{w_0}{HTML}{D8DEE9}
\definecolor{w_1}{HTML}{E5E9F0}
\definecolor{w_2}{HTML}{ECEFF4}

\definecolor{b_ggg}{HTML}{8FBCBB}
\definecolor{b_gg}{HTML}{88C0D0}
\definecolor{b_g}{HTML}{81A1C1}
\definecolor{bb_g}{HTML}{5E81AC}

\definecolor{a_red}{HTML}{BF616A}
\definecolor{a_orange}{HTML}{D08770}
\definecolor{a_yellow}{HTML}{EBCB8B}
\definecolor{a_green}{HTML}{A3BE8C}
\definecolor{a_purple}{HTML}{B48EAD}


% Font
% \renewcommand{\familydefault}{\sfdefault}


% horizontal rule
\newcommand\hr{
    \noindent\rule[0.5ex]{\linewidth}{0.5pt}
}


% hide parts
\newcommand\hide[1]{}


% si unitx
\usepackage{siunitx}
\sisetup{locale = FR}
% \renewcommand\vec[1]{\mathbf{#1}}
\newcommand\mat[1]{\mathbf{#1}}


% tikz
\usepackage{tikz}
\usepackage{tikz-cd}
\usetikzlibrary{intersections, angles, quotes, calc, positioning}
\usetikzlibrary{arrows.meta}
\usepackage{pgfplots}
\pgfplotsset{compat=1.13}


\tikzset{
    force/.style={thick, {Circle[length=2pt]}-stealth, shorten <=-1pt}
}


% theorems
\makeatother
\usepackage{thmtools}
\usepackage[framemethod=TikZ]{mdframed}
\mdfsetup{skipabove=1em,skipbelow=0em}


\theoremstyle{definition}

\newcommand{\declaretheoremstylebox}[2]{
    \declaretheoremstyle[
        headfont=\bfseries\sffamily\color{#1}, bodyfont=\normalfont,
        mdframed={
            linewidth=2pt,
            rightline=false, topline=false, bottomline=false,
            linecolor=#1, backgroundcolor=#1!5
        }
    ]{thm#2box}    
}

\newcommand{\declaretheoremstyleline}[2]{
    \declaretheoremstyle[
        headfont=\bfseries\sffamily\color{#1}, bodyfont=\normalfont,
        mdframed={
            linewidth=2pt,
            rightline=false, topline=false, bottomline=false,
            linecolor=#1
        }
    ]{thm#2line}    
}

\declaretheoremstylebox{b_ggg}{green}
\declaretheoremstylebox{b_g}{blue}
\declaretheoremstylebox{a_red}{red}
\declaretheoremstylebox{a_orange}{orange}
\declaretheoremstylebox{a_yellow!60!a_orange}{yellow}

\declaretheoremstyleline{bb_g}{blue}
\declaretheoremstyleline{b_gg}{green}

\declaretheoremstyle[
    headfont=\bfseries\sffamily\color{a_red}, bodyfont=\normalfont,
    numbered=no,
    mdframed={
        linewidth=2pt,
        rightline=false, topline=false, bottomline=false,
        linecolor=a_red, backgroundcolor=a_red!1,
    },
    qed=\qedsymbol
]{thmproofbox}

\declaretheoremstyle[
    headfont=\bfseries\sffamily\color{b_g}, bodyfont=\normalfont,
    numbered=no,
    mdframed={
        linewidth=2pt,
        rightline=false, topline=false, bottomline=false,
        linecolor=b_g, backgroundcolor=b_g!1,
    },
]{thmexplanationbox}

% \declaretheoremstyle[headfont=\bfseries\sffamily, bodyfont=\normalfont, mdframed={ nobreak } ]{thmgreenbox}
% \declaretheoremstyle[headfont=\bfseries\sffamily, bodyfont=\normalfont, mdframed={ nobreak } ]{thmredbox}
% \declaretheoremstyle[headfont=\bfseries\sffamily, bodyfont=\normalfont]{thmbluebox}
% \declaretheoremstyle[headfont=\bfseries\sffamily, bodyfont=\normalfont]{thmblueline}
% \declaretheoremstyle[headfont=\bfseries\sffamily, bodyfont=\normalfont, numbered=no, mdframed={ rightline=false, topline=false, bottomline=false, }, qed=\qedsymbol ]{thmproofbox}
% \declaretheoremstyle[headfont=\bfseries\sffamily, bodyfont=\normalfont, numbered=no, mdframed={ nobreak, rightline=false, topline=false, bottomline=false } ]{thmexplanationbox}

\declaretheorem[style=thmgreenbox, name=Definition]{definition}
\declaretheorem[style=thmbluebox, numbered=no, name=Example]{example}
\declaretheorem[style=thmorangebox, name=Proposition]{proposition}
\declaretheorem[style=thmredbox, name=Theorem]{theorem}
\declaretheorem[style=thmyellowbox, name=Lemma]{lemma}
\declaretheorem[style=thmredbox, numbered=no, name=Corollary]{corollary}

\declaretheorem[style=thmproofbox, name=Proof]{replacementproof}
\renewenvironment{proof}[1][\proofname]{\vspace{-10pt}\begin{replacementproof}}{\end{replacementproof}}

\declaretheorem[style=thmexplanationbox, name=Proof]{tmpexplanation}
\newenvironment{explanation}[1][]{\vspace{-10pt}\begin{tmpexplanation}}{\end{tmpexplanation}}

\declaretheorem[style=thmgreenline, numbered=no, name=Remark]{remark}
\declaretheorem[style=thmblueline, numbered=no, name=Note]{note}

\newtheorem*{uovt}{UOVT}
\newtheorem*{notation}{Notation}
\newtheorem*{previouslyseen}{As previously seen}
\newtheorem*{problem}{Problem}
\newtheorem*{observe}{Observe}
\newtheorem*{property}{Property}
\newtheorem*{intuition}{Intuition}


\usepackage{etoolbox}
\AtEndEnvironment{vb}{\null\hfill$\diamond$}%
\AtEndEnvironment{intermezzo}{\null\hfill$\diamond$}%
% \AtEndEnvironment{opmerking}{\null\hfill$\diamond$}%

% http://tex.stackexchange.com/questions/22119/how-can-i-change-the-spacing-before-theorems-with-amsthm
\makeatletter
% \def\thm@space@setup{%
%   \thm@preskip=\parskip \thm@postskip=0pt
% }


\newcommand{\exercise}[1]{%
    \def\@exercise{#1}%
    \subsection*{Exercise #1}
}

\newcommand{\subexercise}[1]{%
    \subsubsection*{Exercise \@exercise.#1}
}


\usepackage{xifthen}

% Notes
\usepackage{marginnote}
\let\marginpar\marginnote

\def\testdateparts#1{\dateparts#1\relax}
\def\dateparts#1 #2 #3 #4 #5\relax{
    \marginpar{\small\textsf{\mbox{#1 #2 #3 #5}}}
}

\def\@lecture{}%
\newcommand{\lecture}[3]{
	\ifthenelse{\isempty{#3}}{%
		\def\@lecture{Lecture #1}%
	}{%
		\def\@lecture{Lecture #1: #3}%
	}%
	\section*{\@lecture}
	\marginpar{\small\textsf{\mbox{#2}}}
}


% \renewcommand\date[1]{\marginpar{#1}}


% fancy headers
\usepackage{fancyhdr}
\pagestyle{fancy}

% LE: left even
% RO: right odd
% CE, CO: center even, center odd
\fancyhead[LE, RO]{George C.}

\fancyhead[RO, LE]{\@lecture} % Right odd,  Left even
\fancyhead[RE, LO]{}          % Right even, Left odd
\fancyfoot[RO, LE]{\thepage}  % Right odd,  Left even
\fancyfoot[RE, LO]{}          % Right even, Left odd
\fancyfoot[C]{\leftmark}     % Center

\makeatother


% figure support
\usepackage{import}
\usepackage{xifthen}
\pdfminorversion=7
\usepackage{pdfpages}
\usepackage{transparent}
\newcommand{\incfig}[1]{%
    \def\svgwidth{\columnwidth}
    \import{./figures/}{#1.pdf_tex}
}

% %http://tex.stackexchange.com/questions/76273/multiple-pdfs-with-page-group-included-in-a-single-page-warning
\pdfsuppresswarningpagegroup=1

\setcounter{section}{-1}

\title{Math 1540: HW1}

\begin{document}

\maketitle
\emph{Book Problems:}
\begin{enumerate}
    \item[49.] This is equivalent to proving that a polynomial \(p(x) \in F[x]\) of degree 2 or 3 must have at least one root in \(F\)  iff \(p(x)\) is reducible
    
    Suppose there exists a root, \(a\), in \(F\) then a linear term \((x - a)\) can be divided out from \(p(x)\) (Euclidean division); therefore \(p(x)\) is reducible
    
    Suppose \(p(x)\) is reducible, then \(p(x)\) can be written as \(p(x) = f(x)g(x)\) where \(deg(f(x))\) and \(deg(g(x))\) are not equal to \(0\). \(deg(p(x)) = deg(f(x)) + deg(g(x))\) In the case of \(deg(p(x))= 2\), \(deg(f(x))\) must be less than 2 and not equal to 0, so it's a linear term, and so \(p(x)\) has a root in \(F\). In the case of \(deg(p(x))= 3\), either \(deg(f(x))\) or \(deg(g(x))\) must be less than 2 and not equal to 0, so either \(f(x)\) or \(g(x)\) is a linear term, and so \(p(x)\) has a root in \(F\).

    \item[50.] Since \(p(x) \in F[x]\) is irreducible \(p(x)\) is a maximal ideal therefore either \(g(x) \in (p(x)) \iff p(x)|g(x)\) or \(g(x) \notin (p(x)) \iff (p(x), g(x))\) contains the entire ring \(\iff (p(x), g(x))\) = 1

    \item[63.] Let the monic polynomial \(f(x) = a_0 + a_1 x \ldots a_n x^n\) where \(a_n = 1\) then \(s | 1 \implies s = 1\) and \(r \in \Z\); therefore \(r/s = r\) is an integer

    \item[69.] Using the rational root thm and the fact that this polynomial (degree 3) has a root in \(\Z\) (from Q49). \(1,36,2,18,3,12,4,9,6\) all could be roots. By calculation, we can see that 3 is a root (\(3^3 + 3^2 - 36 = 0\)). Using Euclidean division we find that \(x^3 + x^2 - 36\) splits into \((x-3)(x^2 + 4x - 12)\) Now we must check whether \((x^2 + 4x + 12)\) factors and using the discriminant from the Quadratic formula \(b^2 - 4ac\) we find the discriminant is negative \(16 - 4(1)(12)\), so \((x^2 + 4x + 12)\) cannot be factored further in \(\Q[x]\)  
\end{enumerate}

\emph{Given Problems:}
\begin{enumerate}
    \item[1.] Let \(a = y^3\), then \(y^6 + ry^3 - q^3 / 27\) becomes \(a^2 + ra - q^3 / 27\). Now it is clear that this is in the form of a quadratic and we can solve it using the Quadratic formula. Therefore, \(a = y^3 = (-r \pm \sqrt{r^2 + 4q^3 / 27})/2\) 
    
    \emph{Note: Like in the question I will be using the same notation (variables) as in the textbook. Some of the formulas/equations I use will be from the textbook (Rotman).}

    From page 45 in the textbook we see that \(y^3 z^3 = -q^3 / 27\). From page 46 in the textbook we see that the roots are given as \(y+z, wy + w^2 z, w^2 y +wz\)

    Throughout the entire derivation of the cubic formula in the textbook both \(y\) and \(z\) are interchangeable (there is a symmetry between the two variables)
    
    Let \(y_1 = y^3 =\frac{1}{2}\left(-r + \sqrt{r^2+4 q^3 / 27}\right)\) and \(y_2 = y^3 =\frac{1}{2}\left(-r - \sqrt{r^2+4 q^3 / 27}\right)\) 

    Then
    \begin{eqnarray*}
        y_1y_2 =
        \frac{1}{4}(r^2 - (r^2 + 4q^3 / 27)) =
        \frac{1}{4}(-  4q^3 / 27) =
        - q^3 / 27
    \end{eqnarray*}

    This mirrors an earlier result \(y^3 z^3 = -q^3 / 27\), and suggests that if \(y^3 =\frac{1}{2}\left(-r + \sqrt{r^2+4 q^3 / 27}\right)\) then \(z^3 =\frac{1}{2}\left(-r - \sqrt{r^2+4 q^3 / 27}\right)\) and vice versa. And so, because of the symmetry of the variables \(y\) and \(z\) the choice between the \(+\) and \(-\) does not matter.

    \item[2.] Let the characteristic, \(q = mn \neq 0\), of a field, \(F\), be composite, then \(m1_F \cdot n1_F = q1_F = 0\) but \(m1_F, n1_F \neq 0\); this is a contradiction since this implies that \(m1_f\) and \(n1_F\) are zero divisors, yet are in a field \(F\)
 
    \item[3.] Case \(n = 0\):
    Consider the map \(f: \Z \to F\) such that \(f(m) = m1_F\).
    This map is a homomorphism since:
    \begin{eqnarray*}
        f(m + p) =
        (m + p)1_F =
        m1_F + p1_F =
        f(m) + f(p)
    \end{eqnarray*}
    \begin{eqnarray*}
        f(mp) =
        (mp)1_F =
        m1_F \cdot n1_F =
        f(m) \cdot f(n)
    \end{eqnarray*}
     Using this map we can construct a map to the \(\Q\) similar to how we can create \(\Q\) from \(\Z\) 
    
    Consider the map \(g: \Q \to F\) such that \(g(m/p) = f(m)/f(p)\) with \(p \neq 0\) (to clarify \(m, p \in \Z\) so \(m/p \in \Q\)) This map is a homomorphism since 
    \begin{eqnarray*}
        g(\frac{m_1}{p_1} \cdot \frac{m_2}{p_2}) =
        g(\frac{m_1m_2}{p_1p_2}) =
        (\frac{f(m_1)f(m_2)}{f(p_1)f(p_2)}) =
        (\frac{f(m_1)}{f(p_1)} \cdot \frac{f(m_2)}{f(p_2)}) =
        g(\frac{m_1}{p_1}) \cdot g(\frac{m_2}{p_2})
    \end{eqnarray*}
    \begin{eqnarray*}
        g(\frac{m_1}{p_1} + \frac{m_2}{p_2}) =
        g(\frac{m_1 + m_2}{p_1 + p_2}) =
        (\frac{f(m_1) + f(m_2)}{f(p_1) + f(p_2)}) =
        (\frac{f(m_1)}{f(p_1)} \cdot \frac{f(m_2)}{f(p_2)}) =
        g(\frac{m_1}{p_1}) + g(\frac{m_2}{p_2})
    \end{eqnarray*}

    Since any homomorphism between two fields is injective, this map is an injective map from \(\Q\) to \(F\). Therefore, F contains a subfield isomorphic to image of \(g\), which in this case is \(\Q\). Since the image is in terms of \(1_F\) all fields must contain the image of \(g\) and the image of \(g\) is a subfield, so it must be \(P\)  

    Case \(n = p\):
    When \(n = p\) the image of \(g\) is \(\{0, 1_F \ldots (p-1)1_F \}\) Therefore, \(\Z_p\) is isomorphic to \(P\)    

    \item[4.] We showed in Q63 that if we have a monic polynomial, in this case \(p(x)\), the rational root must be an integer; moreover, since the leading coefficient is 1 (\(s\) must be 1) we know that the rational root \(r/s = r\), and by the rational root thm, \(r | a_0 = 30\). The possible rational roots are \(1,30,2,15,3,10,5,6\). By calculation, can see that \(2\) is a root \(2^3 - 19 + 30\). Using Euclidean division we find that \(x^3 - 19 x + 30\) splits into \((x-2)(x^2 + 2x - 15)\). We know the sum of the other two roots must be \(-2\), the negative of the linear coefficient of \((x^2 + 2x - 15)\) (Note: \((x-a)(x-b) = x^2 - (a+b)x  + ab\) so negative of the linear term is the sum of the roots); therefore, the sum of all the roots is \(s_1 + s_2 + s_3 = 0\).

    In the question, "expression," was somewhat vague so if it meant \(s_1\): by factoring \((x^2 + 2x - 15) = (x + 5)(x - 3)\) (or using the Quadratic formula) we find that \(-5, 3\) are the remaining roots. Since \(-5, 2, 3\) divide \(30\), \(s_1\) divides \(30\). \(s_1 = 3\)     
\end{enumerate}

\end{document}