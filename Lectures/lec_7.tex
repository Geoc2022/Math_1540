\lecture{0}{2023-02-08}{}

\begin{proof}
    To show this let's define a map \(\phi: F[x] \to E\) s.t. \(\phi(f) = f(a)\). The kernel of this map would be \((p(x))\). Because \(E\) is a field then the image of \(phi\) is a domain. Since \(\ker(\phi) = (p(x))\) is prime then \(2 \implies 1, 3\)
    
    Let's define another map \(\tau: F[x]/(p(x)) \to img(\phi)\). Then \(img(\phi) = F(a)\) and so \(\tau(c) = c \forall c \in F\), \(\tau(\bar{x}) = a\).
    
    We know that \(F(a) \subseteq img(\phi)\), so we want to also show \(F(a) \supseteq img(\phi)\) 
    
    If \(F, a \subseteq S\) then all polynomials in \(a\) are in \(S\) Therefore, \(F(a) \supseteq img(\phi)\)  
\end{proof}



\begin{theorem}[existence of splitting field]
    \label{thm:existence of splitting field}
    We know there exists some field such that \(f(x)\) splits. We can write \(f(x) u(x-a_1)\ldots (x-a_n)\). Then \(f(x)\) must split in \(S = F(a_1\ldots a_n)\) 
\end{theorem}


