\lecture{1}{2023-01-25}{}

\section{Motivation}

Galois theory is a branch of mathematics that studies the connection between algebraic equations and their solutions. One of the main motivations for studying Galois theory is the Abel-Ruffini Theorem:

\begin{theorem}[Abel-Rufin]
	\label{thm:abel-rufin}
	There exists a polynomial of degree \(5\) with solutions that are not expressible using algebraic operators (\(+, - , \cdot, /, \sqrt[n]{}\) )
\end{theorem}

\section{Classical Formulas}

Polynomials of degree \([2,4]\) have equations for their solutions.
These equations are often found by reducing a polynomial using a shift:
Given a polynomial: 
\[
	f(x) = a_{n}x^{n} + a_{n-1}x^{n-1} + \ldots a_0  
\] A shift will get rid of the \(x^{n-1}\) term
\[
	f(x - \frac{a_{n-1} }{na_n}) = a_{n}x^{n} + \ldots a_0 
\]
\begin{example}[Quadratic Formula]
	\label{ex:quadratic formula}
	\begin{eqnarray*}
		X^2 + bX +c = 0 \\
		\text{Using a shift of \(-\frac{b}{2}\) removes the linear term:} \\
		(x-\frac{b}{2})^2 + b(x - \frac{b}{2}) + c = 0 \\
		(x^2-bx+\frac{b^2}{4}) + bx - \frac{b^2}{2} + c = 0 \\
		x^2+\frac{b^2}{4} - \frac{b^2}{2} + c = 0 \\
		x^2-\frac{b^2}{4} + c = 0 \\
		x^2= \frac{b^2}{4} - c \\
		x = \pm \sqrt{\frac{b^2}{4} - c} \\
		X = \pm \sqrt{\frac{b^2}{4} - c} - \frac{b}{2} \\
	\end{eqnarray*}
\end{example}





