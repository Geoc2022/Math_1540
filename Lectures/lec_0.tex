\lecture{0}{2023-02-27}{}
 
\subsection{Roots of Unity}
With \(x^n - c\) we can get many roots in terms of each other \(\alpha , \alpha \omega , \alpha \omega ^2 \ldots \alpha \omega ^{n - 1} \)
where \(\omega\) is an n-th root of Unity

The n-th roots of unity always form a cyclic group under multiplication

\begin{theorem}
    If \(n\) is a positive interger
    then \(n - \sum_{d | n} \phi(d)\) 
    where \(\phi\) is the torient funtion  
\end{theorem}

% charterization

\begin{definition}[primitive element]
    \label{def:primitive element}
    Let \(F\) be a finite feild , we say \(a \in F\) is pimative iff \(F = \mathbb{Z} _p (a)\) 
\end{definition}  

