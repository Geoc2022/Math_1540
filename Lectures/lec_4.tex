\lecture{4}{2023-02-01}{}

\begin{definition}[splits]
	\label{def:splits}
	A polynomial, \(f(x)\), splits in \(F[x]\) iff it can be written as a product of linear factors i.e \(F\) contains all the roots of \(f(x)\) 
\end{definition}

\begin{theorem}
	Let \(f(x) \in F[x]\), there exists a field extension \(\frac{E}{F}\)   for which \(f(x)\) splits
\end{theorem}
\begin{proof}
	By induction:
	
	\emph{Base Case:} Trivial
	
	\emph{Inductive Step:} let \(f(x)\) be deg(f) = n + 1
	Write \(f(x) = p(x)g(x)\) \(p(x)\) irreduaclie. Find a field (\(B/F\)) for which \(p(x)\) has a root \(p(x)=(x-a)h(x)\)then \(f(x)=(x-a)h(x)g(x)\) then falls to previous case 
\end{proof}

\begin{note}
	This result is similar to the FTA
\end{note}

\begin{definition}[prime subfield]
	\label{def:prime subfield}
	Let \(F\) be a field, then
	\[
		P = \bigcap_{0 \neq S \in F} S
	\]
	\(P\) be the prime subfield of \(F\)  
\end{definition}
\begin{theorem}
	\(P\) a prime subfield is either isomorphic to \(\Q\) or \(\Z_p\)  
\end{theorem} 
% Add proof from hw
