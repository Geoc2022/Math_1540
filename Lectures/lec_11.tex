\lecture{11}{2023-02-24}{}

If \(f(x) \in F[x]\) is an irreducable polynomial, Galios group acts tansivtivley on the rorots of \(f(x)\)

\begin{example}
    For degree 3 irrreduables there areonly two subgroups (\(S_3, A_3\) ) that act tansitivly
\end{example}

\begin{example}
    For degree 4 irreduables there are isomorphisms \(S_4, A_4, C_4, D_4, V\) where \(V = \{(1), (12)(34), (13)(24), (14)(23)\}\)  
\end{example}

% \(D_4 = \Z_4 semi \Z_2\)

It is clear to see that \(\Gal (E/B) \leq \Gal(E/F)\) given that \(F \subseteq B\subseteq E\)

\begin{lemma}
    Let \(F \subseteq B\subseteq E\) with \(B/F\) as a aplitting feild of polynomial \(f(x) \n F[x]\) then any \(\sigma \in \Gal(E/F), \sigma |_B \in \Gal(B/F)\)   
\end{lemma}
 
\begin{theorem}
    Let \(F \subseteq B\subseteq E\) \(\Gal(E/F) normal \Gal(E/B)\) 
\end{theorem}


